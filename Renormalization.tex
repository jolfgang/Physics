\documentclass[a4paper,10pt]{ltjsarticle} %A4: 21.0 x 29.7cm
\usepackage{amsmath,amsfonts,amssymb}
\usepackage{mathtools}
\usepackage{bm}
\usepackage{ascmac}
\usepackage{fancybox}
\usepackage{graphics}
\usepackage{graphicx,xcolor}
\usepackage{here} %画像の表示位置調整用
\usepackage{type1cm}
\usepackage{hyperref}
\usepackage{enumerate}
\usepackage{braket}

\begin{document}
\title{繰りこみ群
  
%\large{\\副題がある場合はここに記入}
%副題が不要の場合は%\large{}というように%を前につける
}
%\author{}
%\date{\today}}
\maketitle
\section{The renormalization idea}
リノーマライゼーショングループの概念の要約

- **リノーマライゼーショングループの基本概念**
  - 現代の均衡臨界現象へのアプローチを紹介。
  - 「リノーマライゼーショングループ」という用語は厳密には正しくない。
  - 数学的構造がグループの構造を持たない。
  - 量子場理論のリノーマライゼーションとは本質的に異なるが関連はある。
  - 場の理論と無関係な問題にも適用可能。
  - 用語の起源は1960年代の素粒子物理学。

- **リノーマライゼーショングループの名前の由来と誤解**
  - 元々は再正規化された量子電磁力学の高エネルギー挙動に関するもの。
  - K. Wilsonの影響で臨界現象のスケーリング理論に応用が広がった。
  - 定冠詞「the」の使用は誤解を招く。
  - 普遍的な機械のようなものではなく、問題に応じて適応が必要。

- **リノーマライゼーショングループの枠組み**
  - 問題を定義するパラメータをより簡単なセットに再表現する。
  - 長距離物理学や関心のある物理的側面を保持。
  - 例: 臨界現象の短距離自由度の粗視化、流体乱流の大規模擾乱の効果。

- **数学的方程式とリノーマライゼーショングループフロー**
  - 複雑なパラメータ空間でのフローを記述する数学的方程式が必要。
  - これらのフローの研究と物理問題への示唆がリノーマライゼーショングループ理論の本質。

- **実空間リノーマライゼーションの方法**
  - 格子スピンシステムに適用される実空間リノーマライゼーションから始める。
  - 定量的に制御するのが難しいが、スケーリングと普遍性に強力な示唆を与える。

- **第5章の予告**
  - 空間の次元数に関連する小さなパラメータを使った異なるリノーマライゼーショングループを説明予定。
  - 系統的に改善可能な定量的結果を導く方法を紹介予定。

以上がリノーマライゼーショングループの概念に関する要約です【6:0†source】【6:5†source】。


リノーマライゼーショングループの概念

この章では、均衡臨界現象に対する現代的アプローチの基本概念を紹介します。これらの概念は通常「リノーマライゼーショングループ」というタイトルでまとめられていますが、この用語は非常に不適切です。この手法の数学的構造は、厳密な意味でグループの構造を持っているわけではありません。また、量子場理論におけるリノーマライゼーションが本質的な要素ではありませんが、いくつかのリノーマライゼーショングループの定式化とは密接に関連しています。実際、リノーマライゼーショングループの枠組みは場の理論とは無関係な問題にも適用できます。この名前の起源は1960年代の素粒子物理学に遡り、その当時は対称性やグループ理論によって基礎物理学のすべてが説明できるという楽観的な期待がありました。リノーマライゼーショングループのアイデアの最も初期の応用の一つは、再正規化された量子電磁力学の高エネルギー挙動というやや難解な主題でした。K. Wilsonのビジョンによって、これらの手法がFisher、Kadanoff、および他の人々がその後の数十年にわたって形成していた臨界現象のスケーリング理論において、はるかに広い応用範囲を持っていることが認識されました。しかし、その時点で、この名前はこの主題にしっかりと定着していました。

「リノーマライゼーション」や「グループ」という言葉は不適切な用語の例であるだけでなく、それに先立つ定冠詞「the」の使用もさらに混乱を招きます。この定冠詞は、リノーマライゼーショングループが普遍的な機械のようなものであり、どんな問題も処理し、臨界指数の整然とした表を出力するような誤解を与えます。これは全く誤りです。リノーマライゼーショングループは単なる枠組み、一連のアイデアであり、問題の性質に応じて適応させる必要があるのです。特に、リノーマライゼーショングループアプローチが定量的に成功するかどうかは、問題の性質に大きく依存しますが、そのような成功の欠如は、提供される定性的な見解を必ずしも無効にするものではありません。

すべてのリノーマライゼーショングループの研究には、問題を定義するパラメータを他の、より簡単なセットに再表現し、関心のある物理的側面を不変に保つというアイデアが共通しています。これは、臨界現象の問題において短距離自由度の粗視化のように、関心のある長距離物理学を保持する方法であるかもしれません。流体乱流では、大規模な擾乱の効果の修正を表し、これらの変動がどのようにしてより小さな距離スケールに伝わるかに重点を置きます。時間依存の問題では、無秩序相からのクエンチ後の相順序の動力学のように、初期時間履歴を指定するパラメータの時間的進化に対応し、後の時間の特性が変わらないようにします。

動機が何であれ、これらの手法はすべて、ある複雑なパラメータ空間でのリノーマライゼーショングループフローを記述する数学的方程式で終わります。これらのフローの研究と、それらが物理問題について何を示しているかが、リノーマライゼーショングループ理論の本質です。平衡臨界挙動の文脈では、これらのリノーマライゼーショングループの一般的な側面は、格子スピンシステムに適用される実空間リノーマライゼーションの方法に最も直接的に現れます。そして、このリノーマライゼーショングループが実際に作用する例から始めます。これらの実空間手法は、定量的に制御するのが難しいことが判明します。展開できるような小さなパラメータが実際には存在しないためです。しかし、この特徴は、スケーリングと普遍性に対する驚くほど強力な示唆を弱めるものではなく、これらの結果が一般的な性質として現れることを強調したいと思います。後に、第5章では、空間の次元数に関連する小さなパラメータが現れる、やや異なるタイプのリノーマライゼーショングループについて説明し、これを使用して系統的に改善可能な定量的結果を導くことができます【6:0†source】【6:5†source】。


\subsection{Block spin変換}
3.1 ブロックスピン変換

2次元イジングモデルの典型的な状態(磁場がゼロの場合)を、図3.1および図3.2のスナップショットで見てみましょう。これらはコンピュータシミュレーションによって生成されました。最初の図は臨界点付近で撮影され、全てのサイズのスピン(s = -1)のクラスターが見られます。これが解析を難しくする要因です。

さて、画像を少しぼやけさせて、微視的な詳細がよく見えないようにすると仮定します。このデフォーカス(粗視化)を数学的に実現する方法がブロックスピン変換です。具体的には、正方形を3x3のブロックに分け、それぞれのブロックに9つのスピンを含めます。各ブロックに新しい変数s' = +1を割り当て、その値はブロック内のスピンが主に上向きか下向きかを示します。最も簡単な方法は「多数決ルール」を使用することです。すなわち、上向きスピンが多ければs' = +1、逆の場合は-1とします。

この操作が完了し、全体の画像を3倍の線形係数でリスケールすると、ブロックが元の正方形と同じサイズになります。結果として得られる2つ目の画像は、最初の画像と非常によく似ていることがわかります。実際、図3.1bは図3.1aと同じく、臨界イジングモデルにとって同等に確率的な構成です。このブロック化手順を続け、臨界点から始めれば、すべての画像はほぼ同じに見えます。この観察は臨界システムのスケール不変性を示しています。

一方、少し臨界温度を上回る温度(図3.2)から始めると、元のシステムは図3.1aと非常に似ているかもしれませんが、数回の変換後には全く異なる見え方になります。

これらの議論は非常に質的なものですが、それでもなお、くりこみ群アプローチの基本的な基盤です。

---

3.2 一次元イジングモデル

一次元では、上述のブロックスピンくりこみ群を明示的に実行できることがよくあります。例えば、簡単なゼロ磁場イジングスピン鎖を考えてみましょう。ここでは、簡略化されたハミルトニアン $H = -K Σ s_i s_{i+1}$ を用います。このモデルは、転送行列を使用したり、新しい変数$σ_i = s_i s_{i+1}$ を定義したりすることで解くことができます。

説明のため、サイトを図3.3のように3スピンを含むブロックにグループ化することを考えます。以前と同じ多数決ルールを使用することもできますが、解析上は、各ブロックの中央のスピンの票だけをカウントする方がさらに簡単です。これは $s' = s_{central}$ とすることに対応します。この方法が正当化される理由は、非常に低温では、全てのスピンが同じ方向を向く傾向があるためです。

したがって、このくりこみ群変換は各ブロックの端のスピンを追跡し、中央のスピンをそのままにすることに相当します。この手法は一次元で非常によく機能します。例えば、図3.4のように隣接する2つのブロックを考えます。$s_3とs_4$のスピンについては、これらの自由度に関与する分配関数の要素は次のようになります。

\[ e^{K s_2 s_3} e^{K s_3 s_4} e^{K s_4 s_5} \]

これを展開して $s_3$ と $s_4$ の和を取ると、次のようになります。

\[ (cosh K)^3 (1 + 2s_2 s_5) \]

これにより、くりこみ群変換が成功します。


---

この翻訳はブロックスピン変換の一次元および二次元のイジングモデルに関する基本的な説明をカバーしています。

 3.1 ブロックスピン変換 (続き)

次に、ブロックされた系のハミルトニアン \( \mathcal{H}' \) をより具体的に定義します。元の系は次の分配関数によって記述されます。

\[ Z = \text{Tr}_s e^{-\beta \mathcal{H}(s)}, \tag{3.1} \]

ここで、以下では常に因子 \( \beta = \frac{1}{k_B T} \) を \( \mathcal{H} \) 内の交換結合 \( J \) や磁場 \( H \) などのさまざまなパラメータの定義に吸収します。これにより、いわゆる縮約ハミルトニアンが定義されます。我々の多数決ルールは、トレースの下に射影演算子を挿入することによって実装されます。各ブロックについて次のように定義します。

\[ \tau(s'; s) = \begin{cases}
1 & \text{if } \sum_{i \in \text{block}} s_i > 0, \\
0 & \text{otherwise}.
\end{cases} \tag{3.2} \]

新しいハミルトニアンは次のように定義されます。

\[ e^{-\mathcal{H}'(s')} = \text{Tr}_s \prod_{\text{blocks}} \tau(s'; s) e^{-\mathcal{H}(s)}. \tag{3.3} \]

特に重要なのは、次の等式が成り立つことです。

\[ Z = \text{Tr}_{s'} e^{-\mathcal{H}'(s')}, \tag{3.4} \]

すなわち、元の系とブロックされた系の分配関数 \( Z \) は同じです。しかし、上記の変換はそれ以上のものを保持します。式 (3.3) は、ブロックレベルでのスピン \( s', s'', s''' \) などに依存する量の確率分布全体が不変であることを意味します。これにはすべての長波長の自由度が含まれます。したがって、問題の大規模な距離の物理全体が、ブロックスピンまたはリノーマライズスピンではなく、元のスピンまたは裸のスピンに関して表現されるべきという違いを除いて、リノーマライゼーショングループの手続きによって手付かずのまま残されます。

これにより、リノーマライゼーショングループの中心的な仮定である短距離相互作用が優勢であるという仮定が実際にどのように成り立つかがわかります。これが具体的にどのように成り立つかについては、セクション4.3でより詳細に検討しますが、この中心的な仮定の正しさは、スケーリングと普遍性の結果によって最もよく証明されます。これらの結果は、多くの実際の実験や数値シミュレーションでテストされており、また正確に解けるモデルでも検証されています。


\section*{1次元Isingモデルにおける繰りこみ群方程式の導出}

1次元Isingモデルにおける繰りこみ群方程式を導くために、まずモデルのハミルトニアンを定義し、ブロックスピン変換を適用します。

\subsection*{1次元Isingモデルのハミルトニアン}

1次元Isingモデルのハミルトニアンは以下のように表されます:

\[
H = -J \sum_{i} S_i S_{i+1} - H \sum_i S_i
\]

ここで、\( S_i \)はサイト\( i \)のスピン変数(±1)、\( J \)は隣接スピン間の結合定数、\( H \)は外部磁場です。

\subsection*{Decimationによるブロックスピン変換}

Decimation法を用いる場合、3つのスピンのうち中央のスピンのみを保持します。具体的には、元のスピン列 \( S_i \) から、2つ飛ばしで新しいスピン列 \( S'_i \) を構成します。これを用いると、ブロックスピン変換後のハミルトニアンを導きます。

\subsection*{ハミルトニアンの再構成}

元の系において、3つのスピン \( S_i, S_{i+1}, S_{i+2} \) の間の相互作用を考えます。この3スピンに関して、decimation後の有効ハミルトニアンを導出するために、中央のスピン \( S_{i+1} \) を積分します。

元のハミルトニアンの部分和:

\[
H = -J (S_i S_{i+1} + S_{i+1} S_{i+2}) - H (S_i + S_{i+1} + S_{i+2})
\]

これを用いて、中央スピン \( S_{i+1} \) を積分します:

\[
Z = \sum_{S_{i+1}=\pm1} \exp\left[ \beta J S_i S_{i+1} + \beta J S_{i+1} S_{i+2} + \beta H (S_i + S_{i+1} + S_{i+2}) \right]
\]

この分配関数 \( Z \) を計算することで、中央のスピンが積分された後の有効ハミルトニアン \( H' \) を得ます。

\subsection*{有効ハミルトニアンの導出}

\[
Z = \exp\left[ \beta H (S_i + S_{i+2}) \right] \sum_{S_{i+1}=\pm1} \exp\left[ \beta J (S_i + S_{i+2}) S_{i+1} + \beta H S_{i+1} \right]
\]

ここで、sum部分を計算します:

\[
\sum_{S_{i+1}=\pm1} \exp\left[ \beta J (S_i + S_{i+2}) S_{i+1} + \beta H S_{i+1} \right] = 2 \cosh\left[ \beta J (S_i + S_{i+2}) + \beta H \right]
\]

これにより、次のように書けます:

\[
Z = 2 \exp\left[ \beta H (S_i + S_{i+2}) \right] \cosh\left[ \beta J (S_i + S_{i+2}) + \beta H \right]
\]

\subsection*{繰りこみ群方程式の導出}

この結果を用いて、ブロックスピン変換後の有効ハミルトニアンは以下のようになります:

\[
H' = -J' \sum_{i} S'_i S'_{i+1} - H' \sum_i S'_i
\]

ここで、\( J' \) と \( H' \) は次の関係で与えられます:

\[
\exp(-\beta J') = \frac{\cosh(\beta J + \beta H)}{\cosh(\beta H)}
\]

\[
H' = 2H + k_B T \log\left( \cosh(\beta J + \beta H) / \cosh(\beta H) \right)
\]

以上により、1次元Isingモデルにおけるdecimation法を用いた繰りこみ群方程式が導出されました。

\subsection*{参考文献}

この問題に関する詳しい情報は以下の文献にあります:

\begin{itemize}
    \item Kadanoff, L. P. (1966). "Scaling Laws for Ising Models near \( T_c \)". Physics, 2(6), 263-272.
    \item Wilson, K. G. (1971). "Renormalization Group and Critical Phenomena. I. Renormalization Group and the Kadanoff Scaling Picture". Physical Review B, 4(9), 3174-3183.
    \item Plischke, M., \& Bergersen, B. (2006). "Equilibrium Statistical Physics". World Scientific.
\end{itemize}

\end{document}